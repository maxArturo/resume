% resume.tex
%
% (c) 2002 Matthew Boedicker <mboedick@mboedick.org> (original author) http://mboedick.org
% (c) 2003-2007 David J. Grant <davidgrant-at-gmail.com> http://www.davidgrant.ca
% (c) 2010 Max Alcala <maxarturo-at-gmail.com> http://www.maxarturoas.com
%
%This work is licensed under the Creative Commons Attribution-Noncommercial-Share Alike 2.5 License. To view a copy of this license, visit http://creativecommons.org/licenses/by-nc-sa/2.5/ or send a letter to Creative Commons, 543 Howard Street, 5th Floor, San Francisco, California, 94105, USA.

\documentclass[letterpaper,11pt]{article}

%-----------------------------------------------------------
%Margin setup

\setlength{\voffset}{0.1in}
\setlength{\paperwidth}{8.5in}
\setlength{\paperheight}{11in}
\setlength{\headheight}{0in}
\setlength{\headsep}{0in}
\setlength{\textheight}{11in}
\setlength{\textheight}{9.5in}
\setlength{\topmargin}{-0.65in}
\setlength{\textwidth}{7.0in}
\setlength{\topskip}{0in}
\setlength{\oddsidemargin}{-0.25in}
\setlength{\evensidemargin}{-0.25in}

%-----------------------------------------------------------
%\usepackage{fullpage}
%\textheight=9.0in
\usepackage{shading}
\usepackage{hyperref}
\usepackage[utf8]{inputenc}
\pagestyle{empty}
\raggedbottom
\raggedright
\setlength{\tabcolsep}{0in}

%-----------------------------------------------------------
%Custom commands
\newcommand{\resitem}[1]{\item #1 \vspace{-3pt}}
\newcommand{\resheading}[1]{{\large {\textbf{#1 \vphantom{p\^{E}}}}}}
\newcommand{\addline}{\line(1,0){500}}
\newcommand{\ressubheading}[4]{
	\begin{tabular*}{6.5in}{l@{\extracolsep{\fill}}r}
			\textbf{#1} & #2 \\
			\textit{#3} & \textit{#4} \\
	\end{tabular*}\vspace{-6pt}}
%-----------------------------------------------------------


\begin{document}

\begin{tabular*}
	{7in}{l@{\extracolsep{\fill}}r}
	\textbf{
		\Large
		Max Arturo Alcal\'{a} S\'{a}inz}	& 	(+1)773-640-3531 \\
		254 Swett Rd &  	 https://github.com/maxArturo \\
      Woodside, CA 94062 USA				&	  	maxarturo@gmail.com (Twitter: \href{https://twitter.com/MaxArturoAS}{@maxArturoAS})  \\
                              &     https://maxalcala.com \\
	\vspace{2pt}
\end{tabular*}

% \addline
% 
% \begin{center}
% \resheading{Summary}
% 
%   I'm a Software Engineer/Tech Lead with significant buisness and customer-facing experience in both Fortune 50 companies and startups. My current interests cover full-stack development, distributed systems, and the art of fostering a healthy company culture.
% 
% \end{center}
% 
\addline
\vspace{2pt}

%-----------------------------------------------------------
%Skills
\resheading{Destrezas y Habilidades}

\begin{description}
	\item[Lenguajes de programación, frameworks y almacenamiento de datos:]
    TypeScript/ES7/ESNext, Node.js, Kotlin, Ruby/Rails, bases de datos RDBMS/NoSQL. Golang, GraphQL, message queues, scripts de Bash, \LaTeX, y Reactive Extensions (ej Rx.js). Experiencia con programación funcional (Haskell, Elm, módulos funcionales de Typescript).

	\item [Devops/CI:]
    Jenkins/Teamcity, operación via Docker y containerization, Kubernetes, Vagrant, AWS, GCP, instrumentación y análisis de logs, scripting de Bash y administración básica de servidores Linux. Experiencia previa con Terraform/Chef. Testing automatizado via Cypress/Puppeteer.

  \item [Frontend:]
    HTML5, preprocesadores CSS, Javascript/precompilers (Babel/TypeScript), visualización de datos en el browser (d3, crossfilter, SVGs), React.js, Redux, Angular 6/7/8.

	\item[Administración, liderazgo y ambiente empresarial:]
    Experiencia colaborando con todos los niveles de gerencia en cada paso de desarrollo de software (requerimentos, ensayos, pruebas de viabilidad). Excelente habilidad analítica y  de comunicación con distintas audiencias. Experiencia previa trabajando directamente, tanto con VPs de compañías Fortune 50 como usuarios de producto. Experiencia colaborando remoto, con equipos foráneos y directos en oficina.

	\item[Lenguajes naturales:]
		Español nativo, Inglés nativo, Japonés intermedio. 

	\item[Certificationes:]
		AWS Certified Solutions Architect - Associate

\end{description}
\addline

%-----------------------------------------------------------
%Experience
\resheading{Experiencia}
\begin{itemize}

\item
  \ressubheading{PriceWaterhouseCoopers, LLC (PwC)}{San Jose, CA}{PwC Labs, Gerente Senior}{Presente}
	\begin{itemize}
    	\resitem{Desarrollar soluciones de cero para apoyar la gestión de equipos con clientes}
		\resitem{Promover nuevas técnicas y herramientas para optimizar la entrega de proyectos y minimzar costos}
		\resitem{Proveer análisis y arquitectura de sistemas para optimizar la experiencia de los equipos en PwC}
	\end{itemize}

\item
  \ressubheading{\href{https://clearcover.com}{Clearcover.com}}{Chicago, IL}{Ingeniero de Software Principal}{2019 - 2021}
	\begin{itemize}
    	\resitem{Encabezó equipo responsable de desarrollar el software de pólizas de seguros de la compañía}
    	\resitem{Construyó sistema de templetes para generar formas de seguros PDF via tecnologías AWS/Puppeteer/Node }
		\resitem{Entrenar, colaborar y gestionar al equipo de desarrolladores para maximizar su productividad y promover una cultura de software conducente a calidad}
		\resitem{Lista de tecnologías utilizada: Functional Typescript(v3.x), Rails, Kotlin, AWS/K8s, Postgres}
	\end{itemize}

\item
  \ressubheading{PriceWaterhouseCoopers, LLC (PwC)}{Chicago, IL}{Grupo "New Ventures", Manager – Gerente Técnico}{2017 - 2019}
	\begin{itemize}
    \resitem{Proveyó liderazo técnico a lo largo de todo el ciclo de vida de software para multiples productos nuevos}
		\resitem{Responsable de todos los aspectos de ejecución técnicos para llevar a cabo la visión de los productos: arquitectura, desarrollo directo, infraestructura de servidores, calidad del software y uso de técnicas óptimas, etc.}
		\resitem{Gerenció desarrolladores "offshore" remotos para asegurar la calidad del software}
		\resitem{Promovió colaboración de desarrolladores de software modernos (estimados de tiempo en base a los desarrolladores, etc) en PwC}
		\resitem{Lista de tecnologías utilizada: ES6/Typescript, Angular 5, Node.js, GCP Suite, Docker, Kubernetes, Firebase, Jenkins, Postgres}
	\end{itemize}

\item
  \ressubheading{ReviewTrackers}{Chicago, IL}{Ingeniero de Software Senior}{2017}
	\begin{itemize}
    \resitem{Desarrolló nuevas features tanto en el browser como en el servidor para el crecimiento de la base de usuarios}
    \resitem{Automatizó tareas repetitivas para aumentar la productividad del equipo de desarrolladores }
    \resitem{Colaboró y entrenó a desarrolladores menos capacitados}
    \resitem{Realizó análisis de eficiencia del software e implementó optimizaciones correspondientes}
        \resitem{Lista de tecnologías utilizada: React, Golang. Ruby, Solr, Heroku/AWS, React, Node.js, Postgres}
	\end{itemize}

\item
  \ressubheading{Shoppertrak}{Chicago, IL}{Ingeniero de Software Senior/Gerente del equipo de software}{2015 - 2016}
	\begin{itemize}
    \resitem{Responsable de desarrollar y administrar el API Node de la aplicación de reportes de la compañía}
    \resitem{Manejó las tareas de su equipo y gestionó las prioridades con equipos de Producto }
    \resitem{Promovió prácticas óptimas para el desarrollo de software en su equipo y entrenó a los desarrolladores más novatos}
    \resitem{Configuró herramientas CI y servidores para la infraestructura requerida del software}
     \resitem{Lista de tecnologías utilizada: Angular, Node.js, AWS, Redshift/PG, TeamCity}
	\end{itemize}

% \newpage

\item
	\ressubheading{Tukaiz, LLC}{Chicago, IL}{Ingeniero de Software}{Mayo - Oct 2015}
	\begin{itemize}
		\resitem{Dió soporte a aplicaciones de Rails 2.3.8, 3, \& 4 }
		\resitem{Re-escribió código principal de Visual Basic a Ruby}
		\resitem{Re-implementó manejador de datos usando Rails/Sidekiq}
	\end{itemize}

\item 
	\ressubheading{Gearbox Software}{Dallas, TX}{Contratista - Desarrollador de Software}{May - Dec 2015}
	\begin{itemize}
		\resitem{Implementó Sitio de Beta y registro para videojuego "Battleborn" usando Ruby/Rails}
		\resitem{Colaboró en experiencia del usuario y diseño interno del API}
	\end{itemize}



\item
	\ressubheading{PepsiCo - Frito-Lay North America}{Plano, TX}{Desarrollador para Cadena de Suministros y Transportación}{2013 - 2015}
	\begin{itemize}
		\resitem{Generó reportes nacionales para la cadena de suministros de Pepsico (Días de inventario, exactitud de demanda), para Frito-Lay US y Canadá}
		\resitem{Proporcionó mantenimento de scripts para transferencia de datos para toda la cadena de suministros de Frito Lay}
		\resitem{Desarrolló reportes para las bases de dato Oracle de la compañía}
	\end{itemize}

% \item
% 	\ressubheading{Texas Instruments, Inc}{Dallas, TX}{Operations -
% 	Microcontrollers Business Planner}{Jan 2012 - Dec 2012}
% 	\begin{itemize}
% 		\resitem{Managed cradle-to-grave manufacturing support and worldwide
% 		shipment of device portfolio from planning to EOL stages}
% 		\resitem{Harnessed Oracle Business Data Warehouses for Material Master,
% 		shipping, revenue and inventory data mining}
% 	\end{itemize}


\end{itemize}
\addline
\\
\resheading{Educación}
\begin{itemize}
\item
	\ressubheading{Maestría, Supply Chain Management}{GPA: 3.72}
{University of Texas at Dallas}{Dec 2012}

\item
	\ressubheading{Licenciatura: Operations Management, dip. de Ciencias de la Computación}
		{GPA: 3.94}{University of Texas at Dallas: Summa Cum Laude}{Dec 2011}
\end{itemize}

\pagebreak
%-----------------------------------------------------------
%Projects
\addline
\\
\resheading{Proyectos personales}
\begin{description}
	\item [amalgam:] Agregador de noticias extremadamente rápido, sin JS, funcional en hardware de baja capacidad. \href{http://links.maxalcala.com}{link}
	
	\item [jibun (WIP):] Un generador de tablero \href{https://en.wikipedia.org/wiki/Hitori}{Hitori} eficiente en tiempo lineal. Es una mejora en los mejores algoritmos existentes, utilizando un algoritmo de grafo acíclico dirigido para generar cuardos cubiertos, y el algoritmo de cobertura exacta "DLX" de Knuth para generar números, usando React.

  \item [d3 handwriting:] Prueba para recrear letras que asemejan escribir a mano, usando glifos de SVG on d3.js. (en github)
\end{description}
%\newpage

\end{document}
